\section{Non-Technical Summary}
Drones, or Unmanned Aerial Vehicles (UAVs) are a class of robots able to move through the air, often using four propellers giving the name `quadcopters'. Navigating the world is challenging for UAVs, as they do not have the same sensing and processing abilities as humans do. A key task for a UAV is determining where it is in the world in order to navigate to a new, desired location. Many solutions exist for this, including familiar ones such as GPS devices or systems that attempt to replicate human vision in order to track where the robot has moved. Current solutions have drawbacks, including cost, weight, and accuracy. 

Weight and accuracy become significant difficulties as the size of UAVs gets smaller. Smaller UAVs are useful for tasks where a larger UAV may be unable to access the desired space due to its size but where flying through the space is still the best method of completing the tasks. Such tasks might include sewage pipe inspection, agricultural data collection in thick vegetation, and indoor security monitoring. 

In order to complete these tasks, traditionally the UAV has needed to determine where it is in the world, using the previously mentioned localization systems like GPS. This is feasible for larger systems, but very small UAVs cannot lift sensors accurate enough to localize themselves well. This project develops a system where low-quality, lightweight sensors are used on a micro-UAV (27 gram weight) to complete a task which would normally require accurate localization: flying through a door.

To achieve this, two main components are developed. The first is a way to locate where a door is in the low-resolution images available from the micro-UAV's onboard camera. Two different methods of locating the door are created, the first based on more traditional computer vision techniques, the other based on convolutional neural networks, a promising area of network-based image processing that tries to mimic how the human brain processes information. The second major component is a flight controller, which actually tells the micro-UAV where to go based on the location of the door in the current image it sees. Again, two different methods are explored to complete this task. A proportional-derivative controller is a simple system which uses standard control techniques to set the UAV on a course where it will fly through the door. A recurrent neural network uses a network of nodes, again structured to mimic how a brain might deal with information, to get the UAV to fly through the door based on sensor data input. The network learns to do this by being shown hundreds of instances of successful flights and attempts to mimic them. The recurrent neural network also takes advantage of a type of node with memory, which is able to remember past inputs in order to deal with time-dependent quantities like velocities. 

Testing of each implemented method yielded the following results: The convolutional neural network is more accurate and faster than the method based on traditional computer vision techniques. The proportional-derivative controller was more successful at flying the UAV through the door than the recurrent neural network. Neither flight controller was able to achieve a consistently successful flight through the door, but the proportional-derivative controller was able to fly through the door in a small percentage of flights from arbitrary locations.
